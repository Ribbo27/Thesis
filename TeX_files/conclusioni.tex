\chapter{Risultati ottenuti e conclusioni}
In quest'ultimo capitolo vengono presentati i risultati ottenuti e i possibili miglioramenti per un possibile sviluppo futuro del progetto.



\section{Risultati ottenuti}
Il lavoro svolto durante questo progetto di stage ha permesso di ottenere un repository in grado di inviare e ricevere dati multimediali in formato JSON codificati in base64 da e verso un qualsiasi tipo di client, rispettando le direttive delle API sviluppate e descritte nel capitolo 3. 

\subsection{Ricezione dei dati}
Dal punto di vista della ricezione dei dati, il sistema, è in grado di ricevere qualsiasi file multimediale sottoforma di oggetti JSON e salvarli logicamente nel database creato e fisicamente sull'ipotetico server che ospiterà il web service.

\subsection{Invio dei dati}
Per quanto riguarda l'invio dei dati, invece, tramite le query descritte nel Capitolo 3, il sistema può recuperare tutti i dati salvati, genericamente oppure in base al tipo di file. Inoltre, si possono reperire note in base alla loro geolocalizzazione o richiedere files entro una certa distanza dalla posizione dell'utente.

\subsection{Sviluppi futuri}

\pagebreak
\section{Conclusioni}
Dall'analisi del risultato finale ottenuto viene compreso come il lavoro fin'ora prodotto sia funzionante ma migliorabile. 
Ad esempio, si potrebbe implementare la possibilità di gestire l'utenza, quindi aggiungere un servizio di autenticazione che permetta all'utente finale di avere uno storico dei file postati da esso o ancora di poter reperire informazioni riguardo le note postate dagli amici degli utenti. Questo prevede una riprogettazione del modello relazionale, poichè, allo stato attuale non è stato predisposto per questo tipo di funzioni.

Un altro aspetto migliorabile del web service è il lato della sicurezza dei dati, non gestito durante questo progetto.
Molti altri aspetti sono migliorabili o modificabili in base al tipo di client frontend, ancora in fase di progettazione, che verrà sviluppato in futuro.