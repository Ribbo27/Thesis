\chapter{Conclusioni e sviluppi futuri}
In quest'ultimo capitolo vengono presentate le conclusioni e i miglioramenti attuabili per un possibile sviluppo futuro del progetto.

\section{Conclusioni}
Il lavoro svolto durante questo progetto di stage ha permesso di ottenere un repository in grado di ricevere e inviare dati multimediali \textit{\textbf{codificati in Base64}} in \textit{\textbf{formato JSON}} da e verso un qualsiasi tipo di client, rispettando le direttive delle API sviluppate e descritte nella sezione \ref{sec:APIDoc}. \\
Nello specifico, la parte di ricezione dei dati, permette di prendere i files multimediali ricevuti in input, sottoforma di \textit{\textbf{oggetti JSON}}, salvandoli fisicamente sul disco locale della macchina che ospiterà il sistema e creando, a livello logico, i record corrispondenti al file nel database.
La parte di invio dei dati, invece è in grado di interrogare il database in base ai diversi parametri che è possibile passare in input, tramite query string e non, con le \textit{\textbf{richieste HTTP}} di tipo \textit{\textbf{GET}}.

\section{Sviluppi futuri}
Per quanto il sistema sia funzionante, è possibile migliorarlo sotto diversi aspetti.
Riguardo al codice attualmente sviluppato si potrebbe effettuare una fase di \textit{\textbf{refactoring}} in modo tale da alleggerire le operazioni effettuate dai controller, soprattutto nella fase di salvataggio, creando delle classi di appoggio che si occupano di astrarre la logica di memorizzazione.
Relativamente al sistema di archiviazione, invece, si potrebbe implementare la possibilità del salvataggio dei files sul cloud, utilizzando un cloud service come ad esempio \textit{\textbf{Amazon S3}} o \textit{\textbf{Rackspace Cloud Storage}} ben supportati dal framework utilizzato per lo sviluppo del progetto.
Un'altro miglioramento utile e attuabile, non gestito durante lo sviluppo del progetto, è la possibilità di aggiungere un servizio che gestisca l'utenza dell'app introducendo una nuova entità \textit{\textbf{users}} al modello relazionale attuale, quindi implementare un servizio di autenticazione per permettere l'accesso al sistema e dare la possibilità all'utente di avere uno storico delle note postate o anche di reperire informazioni dagli 'amici' degli utenti introducendo una relazione di \textit{\textbf{'friendship'}} tra le istanze dell'entità \textit{\textbf{users}}.
Molti altri aspetti sono migliorabili o modificabili in base al tipo di \textit{\textbf{sistema front-end}}, che al momento è in fase di progettazione e sviluppo.

