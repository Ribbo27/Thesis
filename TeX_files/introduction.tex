\chapter*{Introduzione}



In un mondo sempre più urbanizzato e in cui la popolazione invecchia sempre di più, le città sono un punto di incontro per la società contemporanea. Per questo motivo è sorta la necessità di iniziare a pensare a quali requisiti debba avere una città del futuro. Per venire incontro a questo fenomeno si è introdotto il concetto di \textit{\textbf{walkability}}. Generalmente, il termine walkability indica il livello di sicurezza e comfort per i pedoni in una certa area urbana. Concretamente, una città ha un buon livello di walkability se le strade e i marciapiedi presenti sono in buono stato, se è presente una segnaletica stradale chiara e se sono presenti negozi per ogni tipo di necessità ad una distanza facilmente percorribile a piedi.

Il progetto Urban Story Sharing nasce da un'idea del Dipartimento di Informatica, Sistemistica e Comunicazione dell'Università degli studi di Milano-Bicocca, che ha come obiettivo quello di migliorare la walkability delle aree urbane, attraverso un'applicazione mobile che permette ai cittadini di raccogliere dati relativi ai centri urbani da loro frequentati, descrivendo storie attraverso note, immagini, audio e video. Le annotazioni geolocalizzate verranno salvate su un repostory centralizzato così da poter essere condivise fra utenti. Questo progetto si focalizza, invece, sulla realizzazione di un web service per la raccolta e la distribuzione delle storie urbane, supporterà quindi due delle operazioni \textbf{CRUD} base, quali GET e POST. 
La realizzazione del sistema è stata suddivisa in due parti, lo sviluppo del \textit{\textbf{front-end}} e lo sviluppo del \textit{\textbf{back-end}}.
In questo lavoro di stage si illustrerà la parte di sviluppo del \textit{\textbf{back-end}}.
Con il termine back-end, ci si riferisce ad applicativi software con i quali gli utenti interagiscono indirettamente, solitamente attraverso l'utilizzo di applicazioni front-end.

Nei prossimi capitoli verrà presentato il progetto in ogni sua parte, dalla fase di progettazione a quella di implementazione.  In particolare, nel capitolo 1 si analizzeranno le tecnologie utilizzate per la realizzazione del servizio web. Nel capitolo 2 verrà illustrata la fase di progettazione. Nel capitolo 3 verranno esposte le scelte implementative del progetto. Infine, nel capitolo 4 saranno presentate le conclusioni e gli sviluppi futuri.

