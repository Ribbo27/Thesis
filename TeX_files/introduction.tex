\chapter*{Introduzione}

In un mondo sempre più urbanizzato, le città sono un punto di incontro per la società contemporanea.
Con il termine walkability si definisce quanto è "amichevole" camminare in una certa area. La walkability porta benifici in molti settori come quello sanitario, urbano ed economico.
I fattori che influenzano questa unità di misura includono la presenza o meno di marciapiedi, la qualità degli stessi, nonché le condizioni del traffico e delle strade, modelli di utilizzo del suolo, accessibilità agli edifici e sicurezza.
Inoltre, il concetto di walkability, è molto importante per ottenere un design urbano sostenibile.

Lo scopo del progetto \textbf{Urban Stories Sharing} è la realizzazione di un sistema che permette all'utente di narrare le proprie esperienze urbane descrivendo storie attraverso note, immagini, audio e video. Le annotazioni geolocalizzate verranno salvate su un repostory centralizzato così da poter essere condivise fra utenti. Questo progetto si focalizza, invece, sulla realizzazione di un Web service per la raccolta e la distribuzione delle storie urbane, supporterà quindi due delle operazioni \textbf{CRUD} base, quali GET e POST. 

Con il termine back-end, ci si riferisce ad applicativi software con i quali gli utenti interagiscono indirettamente, solitamente attraverso l'utilizzo di applicazioni front-end.\\
Il progetto Urban Story Sharing nasce da un'idea del Dipartimento di Informatica, Sistemistica e Comunicazione dell'Università degli studi di Milano-Bicocca, che ha come obiettivo quello di migliorare la walkability delle aree urbane, attraverso un'applicazione mobile che permette ai cittadini di raccogliere dati relativi ai centri urbani da loro frequentati.
Grazie alla raccolta di questi dati, si suppone, sia possibile migliorare l'urbanizzazione e quindi lo sviluppo delle città del futuro.//
In questa tesi l'obiettivo è quello di sviluppare un web service per poter, appunto, raccogliere i dati da diversi tipi di dispositivi mobili servendosi del framework Laravel 5.

Nei prossimi capitoli verrà presentato il progetto in ogni sua parte, dalla fase di progettazione a quella di implementazione.  In particolare, nel capitolo 1 si analizzeranno le tecnologie utilizzate per la realizzazione del servizio web. Nel capitolo 2 verrà illustrata la fase di progettazione. Nel capitolo 3 verranno esposte le scelte implementative del progetto. Infine, nel capitolo 4 saranno presentati i risultati ottenuti e le conclusioni.

