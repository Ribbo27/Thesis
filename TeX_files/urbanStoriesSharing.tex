\chapter{Progettazione}

In questo capitolo verrà analizzata la parte progettuale del lavoro svolto focalizzandosi sui due pilastri principali, quali lo schema relazionale nella sezione 2.1 e le API nella sezione 2.2.

\section{Schema Relazionale} 
\subsection{Entità}
Nella progettazione del repository centralizzato, per prima cosa, sono state definite le entità. Un'entità rappresenta una classe di oggetti che hanno proprietà comuni ed esistenza autonoma. Un'occorrenza di queste è una istanza della classe rappresentata dall'entità, dunque l'oggetto stesso. In uno schema, ogni entità ha un nome che la identifica univocamente, e viene rappresentata graficamente tramite un rettangolo con il nome dell'entità al suo interno.
Dai requisiti iniziali bisogna quindi specificare le entità utili allo sviluppo del progetto e i loro relativi attributi. Il modello ER \citep{rif8} finale del progetto è riportato in figura \ref{fig:ERModel};

\begin{figure}[!h]
	\centering
	\includegraphics{schemaER.png}
	\caption{schema relazionale}
	\label{fig:ERModel}
\end{figure}

\pagebreak

\begin{itemize}
	\item \textbf{Photos}
	
		L'entità Photos, tabella \ref{tab:photoModel}, rappresenta, appunto, le foto che vengono memorizzate e richieste nel database. Gli attributi associati sono:
		
		\begin{itemize}
			\item \textbf{id} identificatore numerico univoco.
			\item \textbf{width} rappresenta la larghezza delle foto.
			\item \textbf{height} rappresenta l'altezza delle foto.
		\end{itemize}
	
	\item \textbf{Audios}
	
		La tabella Audios, tabella \ref{tab:audioModel} rappresenta tutti i file di tipo audio. 
		
		Attributi:
		
		\begin{itemize}
			\item \textbf{id} identificatore numerico univoco.
			\item \textbf{duration} rappresenta la durata in secondi.
			\item \textbf{bit\_rate} rappresenta la qualità audio del file caricato.
		\end{itemize}
		
	\item \textbf{Videos}
	
		L'entità Videos, tabella \ref{tab:videoModel}, rappresenta i file video della base di dati.
		Attributi:
		
		\begin{itemize}
			\item \textbf{id} identificatore numerico univoco.
			\item \textbf{duration} rappresenta la durata in secondi.
			\item \textbf{width} rappresenta la larghezza delle foto.
			\item \textbf{height} rappresenta l'altezza delle foto.
			\item \textbf{bit\_rate} rappresenta la qualità audio del file caricato.
			\item \textbf{frame\_rate} rappresenta la qualità video del file.
		\end{itemize}
	
	\item \textbf{Files}
		
		L'entità Files, tabella \ref{tab:fileModel} rappresenta una generalizzazione delle precedenti 3 entità. Gli attributi associati a questa tabella sono quindi tutti quei campi che hanno in comune i file multimediali(foto, audio e video):
		\begin{itemize}
			\item \textbf{id} ovvero un identificatore numerico univoco.
			\item \textbf{size} il quale indica la dimensione del file in byte.
			\item \textbf{format} questo campo indica il formato del file che si vuole caricare o recuperare dal database.
			\item \textbf{path} identifica l'indirizzo in cui viene memorizzato, localmente sul database, il file in questione.
			\item \textbf{created\_at} questo attributo definisce il giorno e l'ora di creazione del file.
			\item \textbf{note\_id} questa è la \textbf{chiave esterna} che associa l'entità \textbf{Files} e l'entità \textbf{Note}  
		\end{itemize}
	Infine ci sono gli ultimi tre attributi, chiavi esterne, \textbf{photo\_id, audio\_id e video\_id }, che contengono l'informazione sul tipo di file, ovvero se sono foto, audio o video, mettendo quindi in relazione la tabella files con la corrispondente tabella. Ovviamente queste tre chiavi posso assumere il valore \textbf{null} poichè, ad esempio, un file non può essere contemporaneamente una foto e un video, o un audio e una foto. 

	\item \textbf{Notes}
	
		L'entità Notes, tabella \ref{tab:noteModel} rappresenta le storie condivise dagli utenti con il sistema. Una nota può contenere più tipi di file.
		
		Attributi: 
		\begin{itemize}
			\item \textbf{id} identificatore numerico univoco.
			\item \textbf{location\_id} chiave esterna che mettere in relazione l'entità Notes con l'entità Locations.
		\end{itemize}
		
	
	\item \textbf{Texts}
	
		L'entità Texts, tabella \ref{tab:textModel}, rappresenta le note di tipo testo condivise dagli utenti. 
		
		Attributi:
		\begin{itemize}
		\item \textbf{id} identificatore numerico univoco.
		\item \textbf{char\_number} numero di caratteri della nota testuale.
		\item \textbf{content} contenuto della nota testo.
		\item \textbf{note\_id} chiave esterna che mette in relazione questa tabella con la tabella Notes.
		\end{itemize}
		
	\item \textbf{Locations}
	
		Infine, l'entità Locations, tabella \ref{tab:locationModel}, rappresenta la geolocalizzazione delle note, infatti, questa, è associata alla tabella Notes.
		
		Attributi:
		\begin{itemize}
			\item \textbf{id} identificatore numerico univoco.
			\item \textbf{latitude} rappresenta la latitudine.
			\item \textbf{longitude} rappresenta la longitudine.
		\end{itemize}
	
\end{itemize}

\subsection{Associazioni}

Le associazioni rappresentano un legame concettuali tra una, due o più entità. In particolare, le associazioni, risultano necessarie per la corretta costruzione e implementazione di un database. In particolare esistono tre tipi di associazioni, \textbf{uno a uno, uno a molti e molti a molti}.

Di seguito verranno illustrate le associazioni che compongono lo schema relazionale di questo progetto.

\begin{itemize}
	\item \textbf{Files -$>$ Photos, Audios, Videos}
	
	La relazione che lega l'entità Files alle entità Photos, Audios e Videos è una associazione di tipo \textbf{uno a uno}. Questo perchè concettualmente un file può essere di un solo tipo per volta. Una foto non può essere al contempo un audio o un video.
	
	\item \textbf{Notes -$>$ Files}
	
	La relazione che lega queste due tabelle è di tipo \textbf{uno a molti} poichè una nota può contenere più files di diverso tipo.
	
	\item \textbf{Notes -$>$ Texts}
	
	Una nota può avere solamente una nota testuale, quindi in questo caso si tratta di associazione \textbf{uno a uno}.
	
	\item \textbf{Locations -$>$ Notes}
	
	In questo caso, può succedere che in uno stesso luogo, utenti diversi, condividano delle note, di conseguenza l'associazione che lega queste tabelle è di tipo \textbf{uno a molti}. 
\end{itemize}


\section{API}

Dopo aver definito la struttura del database, sono state definite le API (\textbf{Application Programming Interface}). Un'API consente a due sistemi di comunicare tra loro. Le API, sostanzialmente, fungono da traduttore per due o più sistemi che interagiscono. Solitamente, ogni API ha documentazione e specifiche che determinano il modo in cui le informazioni possono essere trasferite.
Le API possono utilizzare le richieste HTTP per ottenere informazioni da un'applicazione Web o un server web. Le API sono generalmente classificate come SOAP o REST ed entrambe sono utilizzate per accedere ai servizi Web. In questo progetto sono state definite delle API di tipo REST, che utilizzano gli URL per ricevere o inviare informazioni. REST utilizza quattro diversi verbi HTTP (GET, POST, PUT e DELETE) per eseguire attività.

\subsection{Documentazione API}
\label{sec:APIDoc}


\begin{table}[!h]
\centering
	\begin{tabular}{@{}lll@{}}
		\toprule
		METODO & ENDPOINT          & FUNZIONALITA’                   \\ \midrule
		Get    & /notes            & Lista di oggetti Note           \\
		Get    & /notes/\{id\}     & Recuperare una nota specifica    \\
		Get    & /files/\{id\}     & Recuperare un file specifico     \\
		Get    & /texts/\{id\}     & Recuperare una specifica nota testuale  \\
		Post   & /texts            & Salvare una nuova nota testuale \\
		Get    & /photos/\{id\}    & Recuperare una specifica foto   \\
		Post   & /photos           & Salvare una nuova foto          \\
		Get    & /audios/\{id\}    & Recuperare uno specifico file audio \\
		Post   & /audios           & Salvare un nuovo file audio               \\
		Get    & /videos/\{id\}    & Recuperare un nuovo file video      \\
		Post   & /videos           & Salvare un nuovo video               \\ \bottomrule
	\end{tabular}
\caption{Rappresentazione tabellare degli \textbf{Endpoints} definiti}\label{tab:endpoints}
\end{table}

\subsection{Risorse}

\begin{table}[!h]
\centering
	\begin{tabular}{@{}lll@{}}
		\toprule
		\multicolumn{3}{c}{{ \textbf{NOTE}}}                           \\ \midrule
		\textbf{PARAMETRO} & \textbf{TIPO} & \textbf{DESCRIZIONE}      \\ \midrule
		id                 & int           & Identificatore della nota \\ 
		location           & int           & Chiave Esterna            \\ \bottomrule
	\end{tabular}
\caption{Descrizione risorsa \textbf{Note}}\label{tab:noteModel}
\end{table}

\ \linebreak
\begin{table}[!h]
\centering
	\begin{tabular}{@{}lll@{}}
		\toprule
		\multicolumn{3}{c}{\textbf{LOCATION}}                         \\ \midrule
		\textbf{PARAMETRO}&\textbf{TIPO}&\textbf{DESCRIZIONE}         \\ \midrule
		id        & int     & Indentificatore della geolocalizzazione \\ 
		latitude  & decimal & Rappresentazione della latitudine       \\ 
		longitude & decimal & Rappresentazione della longitudine      \\ \bottomrule
	\end{tabular}
\caption{Descrizione risorsa \textbf{Location}}\label{tab:locationModel}
\end{table}

\ \linebreak
\begin{table}[!h]
\centering
	\begin{tabular}{@{}lll@{}}
		\toprule
		\multicolumn{3}{c}{\textbf{FILE}}                                                                                     \\ \midrule
		\textbf{PARAMETRO} & \textbf{TIPO} & \textbf{DESCRIZIONE} \\ \midrule
		id                                     & int                               & Identificatore del file                  \\ 
		size                                   & float                             & Dimensione                               \\
		format                                 & char                              & Formato                                  \\
		path                                   & mediumtext                        & Percorso di salvataggio                  \\
		created\_at                            & datetime                          & Data e ora di creazione                  \\
		note\_id                               & int                               & Relazione con tabella Notes               \\
		photo\_id                              & int                               & Relazione con tabella Photos              \\
		audio\_id                              & int                               & Relazione con tabella Audios              \\
		video\_id                              & int                               & Relazione con tabella Videos              \\ \bottomrule
	\end{tabular}
\caption{Descrizione risorsa \textbf{File}}\label{tab:fileModel}
\end{table}

\ \linebreak
\begin{table}[!h]
\centering
	\begin{tabular}{@{}lll@{}}
		\toprule
		\multicolumn{3}{c}{\textbf{TEXT}}                                    \\ \midrule
		\textbf{PARAMETRO} & \textbf{TIPO} & \textbf{DESCRIZIONE}            \\
		id                 & int           & Identificatore della nota testo \\
		content            & mediumtext    & Contenuto del testo             \\
		char\_number       & int           & Numero di caratteri             \\
		note\_id           & int           & Relazione con tabella Notes      \\ \bottomrule
	\end{tabular}
\caption{Descrizione risorsa \textbf{Text}}\label{tab:textModel}
\end{table}

\ \linebreak
\begin{table}[!h]
\centering
	\begin{tabular}{@{}lll@{}}
		\toprule
		\multicolumn{3}{c}{\textbf{PHOTO}}                             \\ \midrule
		\textbf{PARAMETRO} & \textbf{TIPO} & \textbf{DESCRIZIONE}      \\
		id                 & int           & Identificatore della foto \\
		width              & int           & Larghezza della foto      \\
		height             & int           & Altezza della foto        \\ \bottomrule
	\end{tabular}
\caption{Descrizione risorsa \textbf{Photo}}\label{tab:photoModel}
\end{table}

\ \linebreak
\begin{table}[!h]
\centering
	\begin{tabular}{@{}lll@{}}
		\toprule
		\multicolumn{3}{c}{\textbf{AUDIO}}                                 \\ \midrule
		\textbf{PARAMETRO} & \textbf{TIPO} & \textbf{DESCRIZIONE}          \\
		id                 & int           & Identificatore del file audio \\
		duration           & int           & Durata dell'audio             \\
		bit\_rate          & int           & Qualità dell'audio            \\ \bottomrule
	\end{tabular}
\caption{Descrizione risorsa \textbf{Audio}}\label{tab:audioModel}
\end{table}

\ \linebreak
\begin{table}[!h]
\centering
	\begin{tabular}{@{}lll@{}}
		\toprule
		\multicolumn{3}{c}{\textbf{VIDEO}}                                 \\ \midrule
		\textbf{PARAMETRO} & \textbf{TIPO} & \textbf{DESCRIZIONE}          \\
		id                 & int           & Identificatore del file video \\
		duration           & int           & Durata del video              \\
		width              & int           & Larghezza del video           \\
		height             & int           & Altezza del video             \\
		bit\_rate          & int           & Qualità dell'audio            \\
		frame\_rate        & int           & Qualità del video             \\ \bottomrule
	\end{tabular}
\caption{Descrizione risorsa \textbf{Video}}\label{tab:videoModel}
\end{table}


\pagebreak
\subsection{Endpoints}

\subsection{GET /notes}
\label{sec:APIDocGetNotes}
Questo endpoint permette di accedere ai dati del database restituendo tutte le note caricate dagli utenti. 
Questo endpoint è accessibile solo con una richiesta GET; inoltre, prevede la possibilità di passare dei parametri in query string.


\begin{table}[!h]
	\centering
	\begin{tabular}{@{}llll@{}}
		\toprule
		\textbf{PARAMETRO} & \textbf{RICHIESTO} & \textbf{VALORE} & \textbf{DESCRIZIONE}                                                                                                                                         \\ \midrule
		lat                & opzionale          & numerico        & \begin{tabular}[c]{@{}l@{}}Latitudine associata alla posizione \\ dell'utente a partire dalla quale \\ si vogliono cercare le note.\end{tabular}                     \\ 
		long               & opzionale          & numerico        & \begin{tabular}[c]{@{}l@{}}Longitudine associata alla posizione \\ dell'utente a partire dalla quale \\ si vogliono cercare le note.\end{tabular}               \\
		max\_distance      & opzionale          & numerico        & \begin{tabular}[c]{@{}l@{}}Distanza massima entro il \\ quale si vogliono cercare le note \\ rispetto alla posizione dell'utente.\end{tabular} \\
		type      		   & opzionale          & text, photo, audio, video& \begin{tabular}[c]{@{}l@{}}Specifica il tipo di note \\ da ricercare.\end{tabular} \\ \bottomrule
	\end{tabular}
\caption{Descrizione dei \textbf{parametri} in input}
\end{table}

\begin{table}[!h]
	\centering
	\begin{tabular}{@{}ll@{}}
		\toprule
		\textbf{PARAMETRO} & \textbf{VALORE}  \\ \midrule
		notes              & array di oggetti Note \\ \bottomrule
	\end{tabular}
\caption{Struttura del \textbf{JSON} restituito con una richiesta \textbf{GET /notes}}
\end{table}

\pagebreak
\begin{lstlisting}[caption = Esempio di richiesta \textbf{GET}, xleftmargin=.3\textwidth, xrightmargin=.3\textwidth]
	/notes
\end{lstlisting}

\begin{lstlisting}[caption = Esempio \textbf{JSON} restituito con  richiesta \textbf{GET /notes}]
[
	{
		"id": 1,
		"location_id": 1,
		"created_at": "2018-10-04 07:18:01"
	},
	{
		"id": 2,
		"location_id": 2,
		"created_at": "2018-10-04 07:20:10"
	},
]


\end{lstlisting}

\pagebreak
\begin{lstlisting}[caption = Esempio di chiamata con \textbf{query string},label={lst:QueryStringLocExample}, xleftmargin=.15\textwidth, xrightmargin=.15\textwidth]
  /notes?lat=12.45&long=13.45&max_distance=1
\end{lstlisting}

\begin{lstlisting}[caption=Esempio \textbf{JSON} restituito con richiesta \textbf{GET /notes} e \textbf{query string}]
{
	"id": 4,
	"location_id": 4,
	"created_at": "2018-10-04 07:21:50",
	"location": {
		"id": 4,
		"latitude": "12.45",
		"longitude": "13.45"
	}
},
{
	"id": 5,
	"location_id": 5,
	"created_at": "2018-10-04 07:23:00",
	"location": {
		"id": 5,
		"latitude": "12.45",
		"longitude": "13.45"
	}
}

\end{lstlisting}


\subsection{GET /notes/\{id\}}
Questo endpoint permette di accedere ai dati del database restituendo la nota specificata tramite parametro. Questo endpoint è accessibile solo con una richiesta GET.


\begin{table}[!h]
\centering
	\begin{tabular}{@{}llll@{}}
		\toprule
		\textbf{PARAMETRO} & \textbf{RICHIESTO} & \textbf{VALORE} & \textbf{DESCRIZIONE}                                                                 \\ \midrule
		id                 & obbligatorio       & intero          & \begin{tabular}[c]{@{}l@{}}Identifica la nota a cui\\ si vuole accedere\end{tabular} \\ \bottomrule
	\end{tabular}
\caption{Descrizione dei \textbf{parametri} in input}
\end{table}



\begin{table}[!h]
	\centering
	\begin{tabular}{@{}ll@{}}
		\toprule
		\textbf{PARAMETRO} & \textbf{VALORE}  \\ \midrule
		note              & oggetto di tipo Note \\ \bottomrule
	\end{tabular}
\caption{Struttura del \textbf{JSON} restituito con richiesta \textbf{GET /notes/\{id\}}}
\end{table}


\begin{lstlisting}[caption=Esempio di richiesta \textbf{GET}, xleftmargin=.3\textwidth, xrightmargin=.3\textwidth]
	/notes/1
\end{lstlisting}

\begin{lstlisting}[caption=Esempio \textbf{JSON} restituito con richiesta \textbf{GET /notes/\{id\}}]
{
	"id": 1,
	"location_id": 1,
	"created_at": "2018-10-04 07:18:01"
}
\end{lstlisting}

\subsection{GET /files/\{id\}}
Questo endpoint permette di accedere ad uno specifico file passando come parametro l'id. Questo endpoint è accessibile solo con una richiesta \textbf{GET}.

\begin{table}[!h]
	\centering
	\begin{tabular}{@{}llll@{}}
		\toprule
		\textbf{PARAMETRO} & \textbf{RICHIESTO} & \textbf{VALORE} & \textbf{DESCRIZIONE}                                                                 \\ \midrule
		id                 & obbligatorio       & intero          & \begin{tabular}[c]{@{}l@{}}identifica il file a cui\\ si vuole accedere\end{tabular} \\ \bottomrule
	\end{tabular}
\caption{Descrizione dei \textbf{parametri} in input}
\end{table}

\begin{table}[!h]
	\centering
	\begin{tabular}{@{}ll@{}}
		\toprule
		\textbf{PARAMETRO} & \textbf{VALORE}  \\ \midrule
		file              & oggetto di tipo File \\ \bottomrule
	\end{tabular}
\caption{Struttura del \textbf{JSON} restituito con richiesta \textbf{GET /files/\{id\}}}
\end{table}


\begin{lstlisting}[caption=Esempio di richiesta \textbf{GET}, xleftmargin=.3\textwidth, xrightmargin=.3\textwidth]
	/files/1
\end{lstlisting}
\pagebreak
\begin{lstlisting}[caption=Esempio \textbf{JSON} restituito con richiesta \textbf{GET /files/\{id\}}]
{
	"id": 1,
	"size": 12345,
	"format": "mp3",
	"path": "audios/audio-1538637710.mp3",
	"note_id": 4,
	"photo_id": null,
	"audio_id": 1,
	"video_id": null,
	"created_at": "2018-10-04 07:21:50"
}
\end{lstlisting}

\subsection{POST /texts}
Questo endpoint permette di salvare note testuali all'interno del database. Non necessita di parametri ed è accessibile con una richiesta HTTP di tipo \textbf{POST}.\\

\begin{table}[!h]
	\centering
	\begin{tabular}{@{}ll@{}}
		\toprule
		\textbf{PARAMETRO} & \textbf{VALORE}  \\ \midrule
		content            & testo\\ 
		latitude           & numerico\\ 
		longitude          & numerico\\ \bottomrule		
	\end{tabular}
\caption{Struttura del \textbf{JSON} da inviare con una \textbf{POST}}
\end{table}

\begin{lstlisting}[caption=Esempio di richiesta \textbf{POST}, xleftmargin=.32\textwidth, xrightmargin=.32\textwidth]
	/texts	
\end{lstlisting}

\begin{lstlisting}[caption=Esempio \textbf{JSON} da inviare con richiesta \textbf{POST /texts}]
{
	"content": "lorem ipsum, quia dolor sit",
	"latitude": 14.6543,
	"longitude": 21.65432
}
\end{lstlisting}

\subsection{GET /texts/\{id\}}
Questo endpoint permette di accedere ad una specifica nota testuale. E' accessibile solo con una richiesta \textbf{GET}.

\begin{table}[!h]
	\centering
	\begin{tabular}{@{}llll@{}}
		\toprule
		\textbf{PARAMETRO} & \textbf{RICHIESTO} & \textbf{VALORE} & \textbf{DESCRIZIONE}                                                                 \\ \midrule
		id                 & obbligatorio       & intero          & \begin{tabular}[c]{@{}l@{}}identifica la nota testuale a cui\\ si vuole accedere\end{tabular} \\ \bottomrule
	\end{tabular}
\caption{Descrizione dei \textbf{parametri} in input}
\end{table}

\begin{table}[!h]
	\centering
	\begin{tabular}{@{}ll@{}}
		\toprule
		\textbf{PARAMETRO} & \textbf{VALORE}  \\ \midrule
		text              & oggetto di tipo Text \\ \bottomrule
	\end{tabular}
\caption{Struttura del \textbf{JSON} restituito con richiesta \textbf{GET /texts/\{id\}}}
\end{table}

\begin{lstlisting}[caption=Esempio di richiesta \textbf{GET}, xleftmargin=.3\textwidth, xrightmargin=.3\textwidth]
	/texts/4
\end{lstlisting}

\begin{lstlisting}[caption=Esempio \textbf{JSON} restituito con richiesta \textbf{GET /texts/\{id\}}]
{
	"id": 4,
	"content": "Testo Di prova TEST RESPONSE",
	"char_number": 28,
	"note_id": 7
}
\end{lstlisting}

\pagebreak
\subsection{POST /photos}
Questo endpoint permette di salvare delle foto all'interno del database. Non necessita di parametri ed è accessibile con una richiesta HTTP di tipo \textbf{POST}. Inoltre il campo \textit{\textbf{image\_data}} deve contenere la \textbf{codifica in base64} del file.\\

\begin{table}[!h]
	\centering
	\begin{tabular}{@{}ll@{}}
		\toprule
		\textbf{PARAMETRO} & \textbf{VALORE}  \\ \midrule
		size               & numerico\\ 
		format             & alfanumerico\\
		width              & numerico\\
		height             & numerico\\
		image\_data        & alfanumerico\\
		latitude           & numerico\\ 
		longitude          & numerico\\ \bottomrule		
	\end{tabular}
\caption{Struttura del \textbf{JSON} da inviare con una \textbf{POST}}
\end{table}

\begin{lstlisting}[caption=Esempio di richiesta \textbf{POST}, xleftmargin=.26\textwidth, xrightmargin=.26\textwidth]
	  /photos	
\end{lstlisting}

\begin{lstlisting}[caption=Esempio \textbf{JSON} da inviare con richiesta \textbf{POST /photos}]
{
	"size" : 4321,
	"format" : "png",
	"width" : 1024,
	"heigth" : 817,
	"image-data" : "iVBORw0KGgoAABBJRU5ErkJggg==",     
	"latitude" : 45.5225169,
	"longitude" : 9.2142299

}
\end{lstlisting}

\pagebreak
\subsection{GET /photos/\{id\}}
Questo endpoint permette di accedere ad una specifica foto presente nel database. E' accessibile solo con una richiesta \textbf{GET}. Tra i valori restituiti il campo \textit{\textbf{image\_data}} contiene la \textbf{codifica in base64} del file.

\begin{table}[!h]
	\centering
	\begin{tabular}{@{}llll@{}}
		\toprule
		\textbf{PARAMETRO} & \textbf{RICHIESTO} & \textbf{VALORE} & \textbf{DESCRIZIONE}                                                                 \\ \midrule
		id                 & obbligatorio       & intero          & \begin{tabular}[c]{@{}l@{}}identifica la foto a cui\\ si vuole accedere\end{tabular} \\ \bottomrule
	\end{tabular}
\caption{Descrizione dei \textbf{parametri} in input}
\end{table}

\begin{table}[!h]
	\centering
	\begin{tabular}{@{}ll@{}}
		\toprule
		\textbf{PARAMETRO} & \textbf{VALORE}  \\ \midrule
		photo              & oggetto photo \\ \bottomrule
	\end{tabular}
\caption{Struttura del \textbf{JSON} restituito con richiesta \textbf{GET /photos/\{id\}}}
\end{table}

\begin{lstlisting}[caption=Esempio di richiesta \textbf{GET}, xleftmargin=.30\textwidth, xrightmargin=.30\textwidth]
	/photos/4
\end{lstlisting}

\begin{lstlisting}[caption=Esempio \textbf{JSON} restituito con richiesta \textbf{GET /photos/\{id\}}]
{
	"id": 4,
	"width": 1920,
	"height": 1080,
	"image_data": "R0lGODlhPQBEAPAwO/AwH+0pCZbEhAAOw==",
	"file": {
		"id": 4,
		"size": 450,
		"format": "gif",
		"path": "photos/image-1538637823.gif",
		"note_id": 6,
		"photo_id": 1,
		"audio_id": null,
		"video_id": null,
		"created_at": "2018-10-04 07:23:43"
	}
}
\end{lstlisting}


\subsection{POST /audios}
Questo endpoint permette di salvare files audio all'interno del database. Non necessita di parametri ed è accessibile con una richiesta HTTP di tipo \textbf{POST}. Inoltre il campo \textit{\textbf{audio\_data}} deve contenere la \textbf{codifica in base64} del file.

\begin{table}[!h]
	\centering
	\begin{tabular}{@{}ll@{}}
		\toprule
		\textbf{PARAMETRO} & \textbf{VALORE}  \\ \midrule
		size               & numerico\\ 
		format             & alfanumerico\\
		duration           & numerico\\
		bit\_rate          & numerico\\
		audio\_data        & alfanumerico\\
		latitude           & numerico\\ 
		longitude          & numerico\\ \bottomrule		
	\end{tabular}
\caption{Struttura del \textbf{JSON} da inviare con una \textbf{POST}}
\end{table}

\begin{lstlisting}[caption=Esempio di richiesta \textbf{POST}, xleftmargin=.26\textwidth, xrightmargin=.26\textwidth]
	  /audios	
\end{lstlisting}

\begin{lstlisting}[caption=Esempio \textbf{JSON} da inviare con richiesta \textbf{POST /audios}]
{
	"size" : 4321,
	"format" : "mp3",
	"duration" : 120,
	"bit_rate" : 12345,
	"audio-data" : "SUQzAwAAA/zw3cGIAAgACAAIAAgACAAAA==",
	"latitude" : 45.5225169,
	"longitude" : 9.2142299
}
\end{lstlisting}

\pagebreak
\subsection{GET /audios/\{id\}}
Questo endpoint permette di accedere ad uno specifico file audio presente nel database ed è accessibile solo con una richiesta \textbf{GET}. Tra i valori restituiti il campo \textit{\textbf{audio\_data}} contiene la \textbf{codifica in base64} del file.

\begin{table}[!h]
	\centering
	\begin{tabular}{@{}llll@{}}
		\toprule
		\textbf{PARAMETRO} & \textbf{RICHIESTO} & \textbf{VALORE} & \textbf{DESCRIZIONE}                                                                 \\ \midrule
		id                 & obbligatorio       & intero          & \begin{tabular}[c]{@{}l@{}}identifica il file audio \\ a cui si vuole accedere\end{tabular} \\ \bottomrule
	\end{tabular}
\caption{Descrizione dei \textbf{parametri} in input}
\end{table}

\begin{table}[!h]
	\centering
	\begin{tabular}{@{}ll@{}}
		\toprule
		\textbf{PARAMETRO} & \textbf{VALORE}  \\ \midrule
		audio              & oggetto audio \\ \bottomrule
	\end{tabular}
\caption{Struttura del \textbf{JSON} restituito con richiesta \textbf{GET /audios/\{id\}}}
\end{table}

\begin{lstlisting}[caption=Esempio di richiesta \textbf{GET}, xleftmargin=.26\textwidth, xrightmargin=.26\textwidth]
           /audios/6
\end{lstlisting}

\begin{lstlisting}[caption=Esempio \textbf{JSON} restituito con richiesta \textbf{GET /audios/\{id\}}]
{
	"id": 6,
	"duration": 10,
	"bit_rate": 4321,
	"audio_data": "SUQzAwAAAs3/cGIAAgACAAIAAgACAAAA==",
	"file": {
		"id": 6,
		"size": 12345,
		"format": "mp3",
		"path": "audios/audio-1538637710.mp3",
		"note_id": 4,
		"photo_id": null,
		"audio_id": 1,
		"video_id": null,
		"created_at": "2018-10-04 07:21:50"
	}
}
\end{lstlisting}

\subsection{POST /videos}
Questo endpoint permette di salvare files video all'interno del database. Non necessita di parametri ed è accessibile con una richiesta HTTP di tipo \textbf{POST}. Inoltre il campo \textit{\textbf{video\_data}} deve contenere la \textbf{codifica in base64} del file.

\begin{table}[!h]
	\centering
	\begin{tabular}{@{}ll@{}}
		\toprule
		\textbf{PARAMETRO} & \textbf{VALORE}  \\ \midrule
		size               & numerico\\ 
		format             & alfanumerico\\
		duration           & numerico\\
		width              & numerico\\
		height             & numerico\\
		bit\_rate          & numerico\\
		frame\_rate        & numerico\\
		video\_data        & alfanumerico\\
		latitude           & numerico\\ 
		longitude          & numerico\\ \bottomrule		
	\end{tabular}
\caption{Struttura del \textbf{JSON} da inviare con una \textbf{POST}}
\end{table}

\begin{lstlisting}[caption=Esempio di richiesta \textbf{POST}, xleftmargin=.26\textwidth, xrightmargin=.26\textwidth]
	   /videos	
\end{lstlisting}

\begin{lstlisting}[caption=Esempio \textbf{JSON} da inviare con richiesta \textbf{POST /videos}]
{
	"size" : 4321,
	"format" : "mkw",
	"duration" : 120,
	"width" : 1080,
	"heigth" : 920,
	"bit_rate" : 12345,
	"frame_rate" : 54321,
	"video_data" : "iVBORw0KGgrkJggg==",    
	"latitude" : 45.5225169,
	"longitude" : 9.2142299
}
\end{lstlisting}

\subsection{GET /videos/\{id\}}
Questo endpoint permette di accedere ad uno specifico file video presente nel database ed è accessibile solo con una richiesta \textbf{GET}. Tra i valori restituiti il campo \textit{\textbf{video\_data}} contiene la \textbf{codifica in base64} del file.

\begin{table}[!h]
	\centering
	\begin{tabular}{@{}llll@{}}
		\toprule
		\textbf{PARAMETRO} & \textbf{RICHIESTO} & \textbf{VALORE} & \textbf{DESCRIZIONE}                                                                 \\ \midrule
		id                 & obbligatorio       & intero          & \begin{tabular}[c]{@{}l@{}}Identifica il file video \\ a cui si vuole accedere.\end{tabular} \\ \bottomrule
	\end{tabular}
\caption{Descrizione dei \textbf{parametri} in input}
\end{table}
\vspace{-6px}
\begin{table}[!h]
	\centering
	\begin{tabular}{@{}ll@{}}
		\toprule
		\textbf{PARAMETRO} & \textbf{VALORE}  \\ \midrule
		video              & oggetto Video \\ \bottomrule
	\end{tabular}
\caption{Struttura del \textbf{JSON} restituito con richiesta \textbf{GET}}
\end{table}
\vspace{-6px}
\begin{lstlisting}[caption=Esempio di richiesta \textbf{GET}, xleftmargin=.26\textwidth, xrightmargin=.26\textwidth]
	  /videos/1
\end{lstlisting}
\vspace{-5px}
\begin{lstlisting}[caption=Esempio \textbf{JSON} restituito con richiesta \textbf{GET /videos/\{id\}}]
{
	"id": 1,
	"duration": 10,
	"width": 1920,
	"height": 1080,
	"bit_rate": 4321,
	"frame_rate": 1234
	"video_data": "ANdFI/UORIffOCHGhRMSxOBv+VSBIAICgAAAKw",
	"file": {
		"id": 2,
		"size": 12345,
		"format": "mp4",
		"path": "videos/video-1538637780.mp4",
		"note_id": 5,
		"photo_id": null,
		"audio_id": null,
		"video_id": 1,
		"created_at": "2018-10-04 07:23:00"
	}
}
\end{lstlisting}
